%% Copernicus Publications Manuscript Preparation Template for LaTeX Submissions
%% ---------------------------------
%% This template should be used for copernicus.cls
%% The class file and some style files are bundled in the Copernicus Latex Package, which can be downloaded from the different journal webpages.
%% For further assistance please contact Copernicus Publications at: production@copernicus.org
%% http://publications.copernicus.org/for_authors/manuscript_preparation.html


%% Please use the following documentclass and journal abbreviations for discussion papers and final revised papers.


%% 2-column papers and discussion papers
\documentclass[essd, manuscript]{copernicus}

%% \usepackage commands included in the copernicus.cls:
%\usepackage[german, english]{babel}
%\usepackage{tabularx}
%\usepackage{cancel}
%\usepackage{multirow}
%\usepackage{supertabular}
%\usepackage{algorithmic}
%\usepackage{algorithm}
%\usepackage{amsthm}
%\usepackage{float}
%\usepackage{subfig}
%\usepackage{rotating}


\begin{document}

\nocite{*}

\title{The AlborEx database: sampling of submesoscale features in the Alboran Sea}

\Author[1]{Charles}{Troupin}
\Author[2]{Ananda}{Pascual}
\Author[2]{Simon}{Ruiz}
\Author[3]{Antonio}{Olita}
\Author[4,5]{Mariona}{Claret}
\Author[2]{Benjamin}{Casas}
\Author[6]{Baptiste}{Mourre}
\Author[7]{Pierre-Marie}{Poulain}
\Author[8]{Antonio}{Tovar-Sanchez}
\Author[9]{Arthur}{Capet}
\Author[2]{Evan}{Mason}
\Author[6]{John~T.}{Allen}
\Author[10]{Amala}{Mahadevan}
\Author[6,2]{Joaqu\'{i}n}{Tintor\'{e}}

\affil[1]{GeoHydrodynamics and Environment Research, University of Li\`{e}ge, Li\`{e}ge, Belgium}
\affil[2]{Instituto Mediterr\'{a}neo de Estudios Avanzados, (CSIC-UIB), Esporles, Spain}
\affil[6]{Balearic Islands Coastal Observing and Forecasting System (SOCIB), Palma de Mallorca, Spain}
\affil[3]{Institute for Coastal Marine Environment-National Research Council (IAMC-CNR) Oristano, Oristano, Italy}
\affil[4]{Joint Institute for the Study of the Atmosphere and Ocean, Seattle, WA, USA}
\affil[5]{Department of Earth and Planetary Sciences, McGill University, Montr\'{e}al, QC, Canada}
\affil[7]{Istituto Nazionale di Oceanografia e di Geofisica Sperimentale (OGS), Trieste, Italy}
\affil[8]{Instituto de Ciencias Marinas de Andaluc\'{i}a, (ICMAN-CSIC), Puerto Real, Spain}
\affil[9]{Modelling for Aquatic Systems (MAST), University of Li\`{e}ge, Li\`{e}ge, Belgium}
\affil[10]{Woods Hole Oceanographic Institution, Woods Hole, MA, USA}

%% The [] brackets identify the author with the corresponding affiliation. 1, 2, 3, etc. should be inserted.

\runningtitle{AlborEx database}
\runningauthor{Troupin et al.}
\correspondence{Charles Troupin (charles.troupin@gmail.com)}



\received{}
\pubdiscuss{} %% only important for two-stage journals
\revised{}
\accepted{}
\published{}

%% These dates will be inserted by Copernicus Publications during the typesetting process.


\firstpage{1}

\maketitle



\begin{abstract}
AlborEx consists of a multi-platform, multi-disciplinary experiment performed in the Alboran Sea (Western Mediterranean Sea) from May 25 to 31, 2014. It included 1 research vessel, 2 gliders, 3 Argo floats and 25 drifters. The objective was to sample the physical and biogeochemical properties of mesoscale and submesoscale features present in the area and observed by mean of satellite images. A particular attention was paid to the vertical motions generated by these features.

The multi-platform approach allows us to collect in situ measurements of the intense front separating the Atlantic and Mediterranean waters present in Eastern Alboran Sea. Surface salinity measured by the R/V highlighted strong gradients, with values ranging from 36.6 (Atlantic Waters) to 38.2 (Mediterranean Waters) over a few kilometers. In addition, 500 samples (chlorophyll and nutrient concentrations) were collected at 66 CTD stations. Near real-time ADCP velocities showed coherent patterns with currents up to 1~m/s in the southern part of the sampled domain. The drifters, released along the frontal zone, followed a massive anticyclonic gyre for a few days. The glider data capture submesocale structures associated with the frontal zone. The of the Argo floats acquired temperature and conductivity profiles while the Prov-bio float also measured oxygen and chlorophyll-a concentrations, colored dissolved organic matter, backscattering at 700 nm, downwelling irradiance at 380, 410, 490 nm, and photosynthetically active radiation (PAR).

\end{abstract}


\introduction 

The variety of physical and biological processes occurring in the ocean at different spatial and temporal scales requires a combination of tools in order to properly understand the underlying mechanisms. Numerical models constitute such a tool: they allows one to design specific experiments that can reproduce some of these processes. Despite the continuous progresses made in models (spatial resolution, parameterization, coupling, \ldots), in situ observation remains an essential ingredient when addressing the complexity of the ocean. 

The perfect observational system would consist in dense array of sensors present at many geographical locations, many depths and measuring almost continuously a wide range of parameters. Obviously such a system is not the reality. Researchers have to rely on the combination of various platforms during a limited period of time, each platform measuring a given set of variables at different spatial and temporal resolutions, spatial coverage, accuracy and depth levels. We will refer to this as multi-platform experiment, by opposition to experiments articulated only around the observations made using a research vessel. 

AlborEx is such a multi-platform experiment. The exercise was performed in the Alboran Sea from from May 25 to 31, 2014, with the objective of getting a deep understanding of the meso and submesoscale processes taking place in this area. The combination of all the platform during that period, described in the next Section, constitutes a particularly rich data set due to the variety of platforms and measured variables concentrated on a relatively small area.

\section{The AlborEx mission}

The particularity of the mission is that the exact sampling area was not known until a few days before the start. Prior to the experiment, satellite images of sea surface temperature (SST) and chlorophyll-a (chla) concentration were acquired in order to provide a general view of the surface features visible in the Alboran Sea. The selected feature of interest was a well-defined front (Fig.~\ref{}) separating Atlantic and Mediterranean waters. 

\subsection{Mission objectives}

Once the area of interest was selected, the main objective was to study the dynamics of the frontal zone and its influence on the biology through a multi-platform, multidisciplinary sampling of the area. The interaction of various scales, in particular mesoscale and submesoscale, makes the observational component of the study particularly challenging. This is why a combination of sensors was deployed simultaneously. 

Of particular interest is the evaluation of the intense vertical motions related to the strong horizontal gradients of sea water properties encountered in the vicinity of the front. These vertical motions 

\subsection{Sampling strategy design}

The section aims to summarize the motivations behind the sampling and deployments. An extensive description of the available data is the object of the next section.

While the remote sensing data represented a meaningful source of information about the front and the small scale features, in situ observation were essential to fulfill the mission objectives. The SOCIB R/V was used to sample the area with vertical profiles acquired though the CTD. Two distinct CTD surveys were performed on a 10 km $\times$ 5 km regular grid. The first survey was run from May 26 to 27 and consists of 34 casts along 5 meridional legs (Fig.~\ref{}). The second survey took place from May 29 to 30 and was made up of 28 casts. These casts were performed at almost similar locations as those of the first survey in order to allow for comparison. 

In addition to the CTDs, the R/V continuously acquired temperature and conductivity along the ship track thanks to the thermosalinograph, from which near surface salinity is derived. Direct measurements of currents were performed during the night (from 10PM UTC to 6AM UTC) and during the 2 CTD surveys, using an acoustic Doppler current profiler (ADCP). The R/V weather station measured air temperature, pressure, wind speed and direction during the whole duration of the mission.

To collect measurements able to address the submesoscale, two gliders were deployed on May 25 inside the study area. The coastal glider carried our measurements up to 200~m depth against 500~m for the deep glider. The horizontal resolution was about 0.5~km for the shallow and 1~km for the deep glider. The initial plan was to have two 50-km meridional tracks, 10 kilometers away one from the other, and to repeat these tracks up to 4 times during the experiment. However, due to the strong eastward currents, different tracks (Fig.~\ref{}) crossing the front several times were made instead.

On May 25, 25 Surface Velocity Program (SVP) drifters were deployed in the frontal area in a tight square pattern (mean distance between drifters around 3~km). Almost all the drifters also provided measurement of the sea water temperature near the surface. On the same day, three Argo floats were programmed to cycle	at intervals ranging from 3 hours to 5 days and to depths up to 2000~m.	

\section{Description of the database}

The AlborEx mission generated a large amount of data in a region scarcely sampled in the past. The synergy between lower-resolution (CTD, drifters) and high-resolution data (ADCP, gliders) makes this dataset unique for the study of submesoscale processes in the Mediterranean Sea. Moreover its multidisciplinary nature makes it suitable to study the interactions between the physical conditions and the biogeochemical variables.

\subsection{Oceanographic context}

The description of features and processes occurring during the mission is not in the scope of this paper. However a brief summary is useful to better understand the general context of the data acquisition. The interested reader is invited to consult \cite{RUIZ2015} for a extensive description of the mission and \cite{PASCUAL2017} for the main findings deduced from the in-situ observations.

Before the campaign, a frontal zone separating warm and cold waters was identified at a longitude between 0 and 1$^{circ}$W. The front was developing around an anticyclonic circulation south of it and displayed filament-like structures (Fig.~\ref{}). During the field experiment, the frontal zone and the associated filaments persisted, with the anticyclonic eddy centered around 36$^{\circ}$30'N, 0.5$^{\circ}$W according to altimetry data. The eddy slowly followed an eastward trajectory in the following days. 

\subsection{File format}

The original data files (i.e. obtained directly from the sensors) are converted to netCDF, a standard format widely adopted in atmospheric and oceanic sciences. Each file contains the measurements acquired by the sensors as well the metadata (mission name, principal investigator, \ldots). The structure of the files follows the Climate and Forecast (CF) conventions and are based on the model of ...

\subsection{Processing levels}

Several processing levels are considered, all of them are provided in the data set, except from the level-0 glider file 

L0 = directly from sensors

L1 = unit conversion applied

L2 = gridded field ; for gliders, means we have profiles

\subsection{Quality control}

Description of the tests and the references.


\conclusions[Conclusions and perspectives]

Unique dataset

Specific conditions with intense front

Potential of the dataset: model validation, mechanism of subduction etc.


\section{Data availability}

SOCIB thredds (thredds = standard)

+ submission to Copernicus Marine Service\\
+ EMODnet Physics\\
+ (if necessary) PANGAEA




\appendix
\section{}    %% Appendix A

\subsection{}     %% Appendix A1, A2, etc.

\appendixfigures  %% needs to be added in front of appendix figures in one-column style (\documentclass[acp, manuscript]{copernicus})

\appendixtables   %% needs to be added in front of appendix tables in one-column style (\documentclass[acp, manuscript]{copernicus})




\authorcontribution{C.T. prepared the figures and the present manuscript.}

\competinginterests{The authors declare that the research was conducted in the absence of any commercial or financial relationships that could be construed as a potential conflict of interest.}

\disclaimer{The authors do not accept any liability for the correctness and appropriate interpretation of the data or their suitability for any use.}

\begin{acknowledgements}
The AlborEx experiment was conducted in the framework of PERSEUS EU-funded project (Grant agreement no: 287600). Glider operations were partially funded by JERICO FP7 project. AP acknowledges support from the Spanish National Research Program (E-MOTION/CTM2012-31014 and PRE-SWOT/CTM2016-78607-P). SR and AP are also supported by the Copernicus Marine Environment Monitoring Service (CMEMS) MedSUB project. EM is supported by a post-doctoral grant from the Conselleria d'Educaci\'{o}, Cultura i Universitats del Govern de les Illes Balears (Mallorca, Spain) and the European Social Fund. AC is a FNRS researcher under the FNRS BENTHOX project (Convention T.1009.15). The profiling floats and some drifters were contributed by the Argo-Italy program. The authors thank A.~Massanet, F.~Margirier, M.~Palmer, C.~Castilla and P.~Balaguer for their efficient work at sea and M.~Menna, G.~Notarstefano and A.~Bussani for their help with the drifter and float data processing.

%NASA Goddard Space Flight Center, Ocean Ecology Laboratory, Ocean Biology Processing Group. Moderate-resolution Imaging Spectroradiometer (MODIS) Terra Ocean Color Data; 2014 Reprocessing. %NASA OB.DAAC, Greenbelt, MD, USA. doi: \doi{10.5067/TERRA/MODIS_OC.2014.0}.
%Accessed on 03/07/2017



\end{acknowledgements}




%% REFERENCES

\bibliographystyle{copernicus}
\bibliography{Alborex2017ESSD.bib}
%%
%% URLs and DOIs can be entered in your BibTeX file as:
%%
%% URL = {http://www.xyz.org/~jones/idx_g.htm}
%% DOI = {10.5194/xyz}


%% LITERATURE CITATIONS
%%
%% command                        & example result
%% \citet{jones90}|               & Jones et al. (1990)
%% \citep{jones90}|               & (Jones et al., 1990)
%% \citep{jones90,jones93}|       & (Jones et al., 1990, 1993)
%% \citep[p.~32]{jones90}|        & (Jones et al., 1990, p.~32)
%% \citep[e.g.,][]{jones90}|      & (e.g., Jones et al., 1990)
%% \citep[e.g.,][p.~32]{jones90}| & (e.g., Jones et al., 1990, p.~32)
%% \citeauthor{jones90}|          & Jones et al.
%% \citeyear{jones90}|            & 1990



%% FIGURES

%% We suggest to put the figures in the correct order and placement within the text. This aids readability.
%% When figures and tables are placed at the end of the MS (article in one-column style), please add \clearpage
%% between bibliography and first table and/or figure as well as between each table and/or figure.


%% ONE-COLUMN FIGURES

%%f
%\begin{figure}[t]
%\includegraphics[width=8.3cm]{FILE NAME}
%\caption{TEXT}
%\end{figure}
%
%%% TWO-COLUMN FIGURES
%
%%f
%\begin{figure*}[t]
%\includegraphics[width=12cm]{FILE NAME}
%\caption{TEXT}
%\end{figure*}
%
%
%%% TABLES
%%%
%%% The different columns must be seperated with a & command and should
%%% end with \\ to identify the column brake.
%
%%% ONE-COLUMN TABLE
%
%%t
%\begin{table}[t]
%\caption{TEXT}
%\begin{tabular}{column = lcr}
%\tophline
%
%\middlehline
%
%\bottomhline
%\end{tabular}
%\belowtable{} % Table Footnotes
%\end{table}
%
%%% TWO-COLUMN TABLE
%
%%t
%\begin{table*}[t]
%\caption{TEXT}
%\begin{tabular}{column = lcr}
%\tophline
%
%\middlehline
%
%\bottomhline
%\end{tabular}
%\belowtable{} % Table Footnotes
%\end{table*}
%
%
%%% MATHEMATICAL EXPRESSIONS
%
%%% All papers typeset by Copernicus Publications follow the math typesetting regulations
%%% given by the IUPAC Green Book (IUPAC: Quantities, Units and Symbols in Physical Chemistry,
%%% 2nd Edn., Blackwell Science, available at: http://old.iupac.org/publications/books/gbook/green_book_2ed.pdf, 1993).
%%%
%%% Physical quantities/variables are typeset in italic font (t for time, T for Temperature)
%%% Indices which are not defined are typeset in italic font (x, y, z, a, b, c)
%%% Items/objects which are defined are typeset in roman font (Car A, Car B)
%%% Descriptions/specifications which are defined by itself are typeset in roman font (abs, rel, ref, tot, net, ice)
%%% Abbreviations from 2 letters are typeset in roman font (RH, LAI)
%%% Vectors are identified in bold italic font using \vec{x}
%%% Matrices are identified in bold roman font
%%% Multiplication signs are typeset using the LaTeX commands \times (for vector products, grids, and exponential notations) or \cdot
%%% The character * should not be applied as mutliplication sign
%
%
%%% EQUATIONS
%
%%% Single-row equation
%
%\begin{equation}
%
%\end{equation}
%
%%% Multiline equation
%
%\begin{align}
%& 3 + 5 = 8\\
%& 3 + 5 = 8\\
%& 3 + 5 = 8
%\end{align}
%
%
%%% MATRICES
%
%\begin{matrix}
%x & y & z\\
%x & y & z\\
%x & y & z\\
%\end{matrix}
%
%
%%% ALGORITHM
%
%\begin{algorithm}
%\caption{�}
%\label{a1}
%\begin{algorithmic}
%�
%\end{algorithmic}
%\end{algorithm}
%
%
%%% CHEMICAL FORMULAS AND REACTIONS
%
%%% For formulas embedded in the text, please use \chem{}
%
%%% The reaction environment creates labels including the letter R, i.e. (R1), (R2), etc.
%
%\begin{reaction}
%%% \rightarrow should be used for normal (one-way) chemical reactions
%%% \rightleftharpoons should be used for equilibria
%%% \leftrightarrow should be used for resonance structures
%\end{reaction}
%
%
%%% PHYSICAL UNITS
%%%
%%% Please use \unit{} and apply the exponential notation


\end{document}
